\unorderedsection{Заключение}

В ходе выполнения работы был разработан прототип приложения,
позволяющего инспектировать информационные модели в среде виртуальной реальности. 
Все задачи проекта были выполнены, а именно:

\begin{enumerate}
    \item реализован процесс извлечения атрибутивной информации модели
    на уровне категорий объектов, задаваемых в Revit;
    \item реализованна автоматическая конверсия информационной модели
    из формата редактора Revit в пакет, загружаемый на клиентской части приложения;
    \item реализована возможность манипулирования представлением информационной модели
    в клиентской части приложения.
\end{enumerate}

Разработанный прототип в дальнейшем может быть модернизирован
или полностью пересмотрен для добавления новой функциональности.
На данный момент рассматривается возможность синхронизации изменений
информационной модели в реальном времени без повторного конвертирования,
что потребует значительного пересмотра работы серверной части приложения.
Помимо этого проводятся исследования по повышению качества
извлечения атрибутивной информации модели.
