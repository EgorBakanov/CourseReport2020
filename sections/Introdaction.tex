\unorderedsection{Введение}
BIM~--~понятие, под которым подразумевают цифровой проект здания или другого объекта инфраструктуры,
которая связана с базой данных всех его физических и функциональных характеристик,
содержащей подробную информацию обо всех элементах модели:
элемент может содержать информацию о габаритах, поставщике и даже серийном номере. 
Изменения в любом элементе системы здания способны повлечь автоматические изменения  параметров и объектов, 
вплоть до изменения чертежей, визуализаций, спецификаций, календарного графика и сметы.
BIM~--~это общий ресурс знаний для получения информации об объекте,
который служит надежной основой для принятия решений в течение всего жизненного цикла
начиная с самой ранней концепции до сноса.
\cite{NationalBIMfaqs}
11 июня 2016 года был утвержден список поручений Правительству Российской Федерации,
направленный на развитие правовой базы использования информационного моделирования в сфере строительства.
\cite{KremlinInstraction2016}

Информационное моделирование является комплексным процессом,
требующим определенной компетенцией в этой области.
Для использования BIM-методологии необходимы навыки использования специализированного программного обеспечения,
коих может быть лишена значительная часть проектной команды.
Для обычных людей крайне сложно воспринимать весь объем информации, закладываемой в BIM.

В связи с развитием технологий в последнее десятилетие
произошел стремительный рост популярности технологии виртуальной реальности.
\cite{Cipresso2018}
Как показывают многочисленные исследования,
использование технологий виртуальной и дополненной реальности может улучшить
производительность при валидации и верификации разрабатываемой модели.
Применение технологии VR способно значительно повысить презентационные качества модели,
что усилит вовлеченность в проект участников,
не имеющих специальных профильных навыков.
\cite{Akpan2018}
Исходя из этого было принято решение о разработке приложения,
способного визуализировать трехмерную репрезентацию информационной модели в VR-среде. 

{\bfЦель работы}~--~разработать прототип приложения, позволяющего инспектировать BIM модели в виртуальной реальности.

\paragraph{Задачи}
\begin{enumerate}
    \item реализовать извлечение атрибутивной информации модели;
    \item реализовать серверную часть приложения, занимающуюся хостингом и предобработкой моделей;
    \item автоматизировать перенос моделей из сред разработки в приложение.
    \item реализовать модуль взаимодействия пользователя с моделью на клиентской части приложения;
\end{enumerate}

Стоит отметить, что данный проект разрабатывается командой из нескольких человек,
поэтому в ходе работы не будут представлены те части,
в которых автор не принимал непосредственного участия при разработке.