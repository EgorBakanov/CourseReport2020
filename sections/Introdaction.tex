\unorderedsection{Введение}
BIM - понятие, под которым подразумевают цифровой проект здания или другого объекта инфраструктуры,
которая связана с базой данных всех его физических и функциональных характеристик,
содержащей подробную информацию обо всех элементах модели:
элемент может содержать информацию о габаритах, поставщике и даже серийном номере. 
Изменения в любом элементе системы здания способны повлечь автоматические изменения  параметров и объектов, 
вплоть до изменения чертежей, визуализаций, спецификаций, календарного графика и сметы.
BIM - это общий ресурс знаний для получения информации об объекте,
который служит надежной основой для принятия решений в течение его жизненного цикла
начиная с самой ранней концепции до сноса.
\cite{nationalBIMfaqs}
11 июня 2016 года был утвержден список поручений Правительству Российской Федерации,
направленный на развитие правовой базы использования
информационного моделирования зданий в сфере строительства.
\cite{kremlinInstraction2016}

Информационное моделирование является комплексным процессом, требующим определенной компетенцией в этой области. 
Для обычных людей крайне сложно воспринимать весь объем информации, закладываемой в BIM.

Поэтому было принято решение перенести презентацию проектов,
созданных на основе технологий BIM, в более удобную и интересную для восприятия среду. 
Визуализации, создаваемые на основе виртуальной реальности,
способны сделать процесс инспектирования более интерактивным и иммерсивным,
при этом скрывая от пользователя колоссальные объемы ненужной ему информации.

Цель работы — разработать приложение, позволяющее инспектировать BIM модели в виртуальной реальности.

\paragraph{Задачи}
\begin{enumerate}
    \item реализовать модуль взаимодействия пользователя с моделью на клиентской части приложения;
    \item реализовать серверную часть приложения, занимающуюся хостингом моделей;
    \item реализовать серверную часть приложения, занимающуюся предобработкой модели;
    \item реализовать извлечение атрибутивной информации модели;
    \item автоматизировать перенос моделей из сред разработки в приложение.
\end{enumerate}

\comment{
\paragraph{Используемый технологический стек}
\begin{itemize}
    \item Unity — межплатформенная среда разработки интерактивных графических приложений,
    используется для разработки клиентской части приложения;
    \item C\# — объектно-ориентированный язык программирования, используемый в клиентской и серверной частях приложения;
    \item ShaderLab — язык программирования шейдеров;
    \item HTC Vive — шлем виртуальной реальности, используемый для тестирования приложения.
\end{itemize}

\paragraph{Функциональные требования}
\begin{itemize}
    \item выбор BIM модели;
    \item управление масштабом модели;
    \item управление отображением различных слоев модели;
    \item загрузка модели на сервер(только для BIM-разработчиков).
\end{itemize}

\paragraph{Нефункциональные требования}
\begin{itemize}
    \item автоматическая загрузка модели в приложение с сервера;
    \item  приемлемая производительность приложения, когда в кадре находится вся модель целиком.
\end{itemize}
}