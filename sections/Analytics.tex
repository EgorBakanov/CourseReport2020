\section{Аналитика}

Данный раздел содержит обзор существующих решений,
направленных на визуализацию информационных моделей в виртуальной реальности.
В ходе их анализа были выявлены функциональные и нефункциональные требования
к реализации системы, приведенные далее. 

\subsection{Существующие решения}
В ходе изучения существующих решений был обнаружен ряд продуктов
как в индустриальной, так и в академической среде.
Ниже приведены несколько примеров,
на основе которых были сформулированы требования к разрабатываемому прототипу.

\paragraph{Индустриальная среда}

\lipsum[2][1-3]
\cite{UnityReflect}

\lipsum[4][1-4]
\cite{IrisVR}

\lipsum[3][1-4]
\cite{Enscape}

\paragraph{Академическая среда}

\lipsum[5][1-4]
\cite{Davidson2019}

\lipsum[6][1-4]
\cite{Sampaio2018}

\lipsum[7][1-8]
\cite{PourRahimian2019}

%\lipsum[2]

\subsection{Требования к системе}
\paragraph{Функциональные требования}
\begin{itemize}
    \item выбор BIM модели;
    \item управление масштабом модели;
    \item управление отображением различных слоев модели;
    \item загрузка модели на сервер(только для BIM-разработчиков).
\end{itemize}

\paragraph{Нефункциональные требования}
\begin{itemize}
    \item автоматическая загрузка модели в приложение с сервера;
    \item приемлемая производительность приложения, когда в кадре находится вся модель целиком.
\end{itemize}