\section{Аналитика}

Данный раздел содержит обзор существующих решений,
направленных на визуализацию информационных моделей в виртуальной реальности.
В ходе их анализа были выявлены функциональные и нефункциональные требования
к реализации системы. 

\subsection{Существующие решения}
В ходе изучения существующих решений был обнаружен ряд продуктов
как в индустриальной, так и в академической среде.
Ниже приведены несколько примеров,
на основе которых были сформулированы требования к разрабатываемому прототипу.

\paragraph{Индустриальная среда}

Unity Reflect -- приложение разрабатываемое компанией Unity Technologies,
на основе их игрового фреймворка Unity.
\cite{UnityReflect}
Reflect обладает интеграцией с несколькими программами информационного моделирования,
такими как Revit, Rhino и Sketchup.
Reflect способен синхронизировать изменения информационной модели с
ее VR отображением в реальном времени, а также извлекать атрибутивную информацию.
К сожалению на момент разработки нашего решения и написания этой статьи
Unity Reflect был не завершенным продуктом, находящимся в активной разработке.

Prospect -- аналогичное приложение от компании IrisVR,
поддерживающее информационные модели
из Navisworks, Revit, Rhino, Sketchup, а также 3D форматы FBX и OBJ.
\cite{IrisVR}
Prospect способен извлекать атрибутивную информацию,
имеет достаточно высокую производительность,
а также возможность проведения многопользовательских сессий.

Enscape -- ещё одно приложение с интеграцией
с Revit, Sketchup, Rhino, ArchiCAD и Vectorworks.
\cite{Enscape}
Enscape может синхронизировать изменения информационной модели в реальном времени
и извлекать атрибутивную информацию.
Из особенностей Enscape стоит отметить широкий набор
настроек и эффектов отображения информационной модели.

\paragraph{Академическая среда}

В статье Джордана Дэвидсона и др. описана разработка прототипа приложения
на основе Enscape расширения для Revit, описанного раннее.
\cite{Davidson2019}
Целью работы было расширение уже имеющихся возможностей Enscape
для экспериментальной проверки пользовательского опыта при инспектировании модели.
Приложение Дэвидсона отличается от предыдущих тем,
что обладает обратной связью с отображаемой информационной моделью,
позволяющей пользователю прямо внутри VR симуляции изменять окружение.
В рамках прототипа эта особенность ограничивалась изменением
внутреннего интерьера, мебели и характеристик окон.
Как утверждает автор, данный подход повышает вовлеченность
клиента в проект на ранних стадиях разработки,
повышает его осведомленность о решениях, принятых архитекторами,
и снижает риск изменений в ``последнюю минуту'',
повышающих стоимость реализации проекта.

%\lipsum[6][1-4] % TO DO
Ещё одно возможное решение задачи описал в своей работе Фарзад Пур Рахимян и др.
\cite{PourRahimian2019}
Помимо частичной обратной связи, аналогичной той,
что представлена в работе Джордана Дэвидсона,
прототип Рахимяна уникален использованием в качестве целевого формата информационных моделей
Industry Foundation Classes -- открытый, международный и
независимый от других производителей стандарт,
разработанный buildingSMART.
\cite{BuildingSmartIFC}
В работе также описана клиент-серверная архитектура решения:
на клиентской части приложения происходит интерактивная демонстрация модели
в виртуальной или дополненной реальности;
сервер же является промежуточным слоем между исходной информационной моделью и визуализацией,
отслеживающим изменения и производящим синхронизацию данных.

%\lipsum[2]

\subsection{Требования к системе}

Далее приведен формализованный список требований к разрабатываемому прототипу.

\paragraph{Функциональные требования}
\begin{itemize}
    \item выбор BIM модели;
    \item управление масштабом модели;
    \item управление отображением различных слоев модели;
    \item загрузка модели на сервер(только для BIM-разработчиков).
\end{itemize}

\paragraph{Нефункциональные требования}
\begin{itemize}
    \item автоматическая загрузка модели в приложение с сервера;
    \item совместимость с форматом Revit,
    как наиболее распространенной системы информационного моделирования;
    \item предварительная обработка информационной модели
    для повышения производительности визуализации.
\end{itemize}