\section{Аналитика}

Данный раздел содержит обзор существующих решений,
направленных на визуализацию информационных моделей в виртуальной реальности.
В ходе их анализа были выявлены функциональные и нефункциональные требования
к реализации системы. 

\subsection{Существующие решения}
В ходе изучения существующих решений был обнаружен ряд продуктов
как в индустриальной, так и в академической среде.
Ниже приведены несколько примеров,
на основе которых были сформулированы требования к разрабатываемому прототипу.

\paragraph{Индустриальная среда}

Unity Reflect -- приложение разрабатываемое компанией Unity Technologies,
на основе их игрового фреймворка Unity.
Reflect обладает интеграцией с несколькими программами информационного моделирования,
такими как Revit, Rhino и Sketchup.
Reflect способен синхронизировать изменения информационной модели с
ее VR отображением в реальном времени, а также извлекать атрибутивную информацию.
К сожалению на момент разработки нашего решения и написания этой статьи
Unity Reflect был не завершенным продуктом, находящимся в активной разработке.
\cite{UnityReflect}

Prospect -- аналогичное приложение от компании IrisVR,
поддерживающее информационные модели
из Navisworks, Revit, Rhino, Sketchup, а также 3D форматы FBX и OBJ.
Prospect способен извлекать атрибутивную информацию,
имеет достаточно высокую производительность,
а также возможность проведения многопользовательских сессий.
\cite{IrisVR}

Enscape -- ещё одно приложение с интеграцией
с Revit, Sketchup, Rhino, ArchiCAD и Vectorworks.
Enscape может синхронизировать изменения информационной модели в реальном времени
и извлекать атрибутивную информацию.
Из особенностей Enscape стоит отметить широкий набор
настроек и эффектов отображения информационной модели.
\cite{Enscape}

\paragraph{Академическая среда}

В статье Джордана Дэвидсона и др. описана разработка прототипа приложения
на основе Enscape расширения для Revit, описанного раннее.
Целью работы было расширение уже имеющихся возможностей Enscape
для экспериментальной проверки пользовательского опыта при инспектировании модели.
Приложение Дэвидсона уникально тем, что обладает обратной связью
с отображаемой информационной моделью,
позволяющей пользователю прямо внутри VR симуляции изменять окружение.
В рамках прототипа эта особенность ограничивалась изменением
внутреннего интерьера, мебели и характеристик окон.
Как утверждает автор, данный подход повышает вовлеченность
клиента в проект на ранних стадиях разработки,
повышает его осведомленность о решениях, принятых архитекторами,
и снижает риск изменений "в последнюю минуту",
повышающих стоимость реализации проекта.
\cite{Davidson2019}

\lipsum[6][1-4] % TO DO
\cite{PourRahimian2019}

%\lipsum[2]

\subsection{Требования к системе}
\paragraph{Функциональные требования}
\begin{itemize}
    \item выбор BIM модели;
    \item управление масштабом модели;
    \item управление отображением различных слоев модели;
    \item загрузка модели на сервер(только для BIM-разработчиков).
\end{itemize}

\paragraph{Нефункциональные требования}
\begin{itemize}
    \item автоматическая загрузка модели в приложение с сервера;
    \item приемлемая производительность приложения, когда в кадре находится вся модель целиком.
\end{itemize}