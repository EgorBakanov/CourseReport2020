\subsection{Оптимизационные подходы}

Теперь можно рассмотреть подходы повышения производительности визуализации.
Важность этого аспекта была выявлена 
ещё на самых ранних стадиях разработки прототипа.
Информационные модели зачастую обладают очень комплексной геометрией.
Более того, визуализация в виртуальной реальности накладывает
дополнительные ограничения, так как изображение проходит двукратную отрисовку
из-за наличия двух глаз с разной перспективой.
В добавок к этому низкая частота кадров в виртуальной реальности
может вызывать плохое самочувствие у пользователя.
\cite{Weech2019}

\paragraph{Причины проблем с производительностью}

%\lipsum[6]

\paragraph{Существующие подходы}

%\lipsum[6]

\paragraph{Выбранный подход}

%\lipsum[6]

\comment{
    TODO:

    - вступление
        можно написать о важности оптимизации,
        т.к. при больших/сложных моделях фпс падает в нулину
    - Причины проблем с производительностью
        много полигонов
        много draw call'ов
    - Существующие подходы
        LOD
        Batching
        Culling
    - Реализуемый подход
        описать, что я слепливаю меши через 3ds max 

    тут можно набрать инфы
        https://docs.google.com/document/d/1UFfMFDzLpAvTebCliWfk5ZwtrGVYgcodpkItyuvP6hA/edit
}
