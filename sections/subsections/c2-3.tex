\subsection{Обзор оптимизационных подходов}
\label{subsections:Optimization}

Теперь можно рассмотреть подходы повышения производительности визуализации.
Важность этого аспекта была выявлена 
ещё на самых ранних стадиях разработки прототипа.
Информационные модели зачастую обладают очень комплексной геометрией.
Более того, визуализация в виртуальной реальности накладывает
дополнительные ограничения, так как изображение проходит двукратную отрисовку
из-за наличия двух глаз с разной перспективой.
В добавок к этому низкая частота кадров в виртуальной реальности
может вызывать плохое самочувствие у пользователя.%
\cite{Weech2019}

\paragraph{Вспомогательные понятия}

\begin{rglossary}
    Для начала введем несколько вспомогательных терминов:

    \rglossarylineExpl{Draw call}
    {рус. Вызов отрисовки}
    {команда отрисовки полигональной сетки,
    отдаваемая центральным процессором графическому.}

    \rglossarylineExpl{Шейдер}
    {англ. shader}
    {разновидность компьютерных программ, запускаемых на графических процессорах,
    предназначенных для отрисовки изображений.}

    \rglossarylineExpl{Меш или полигональная сетка}
    {англ. polygon mesh}
    {структура данных, содержащая набор вершин, ребер и граней,
    определяющих форму многогранного объекта.}

    \rglossaryline{Графический материал}
    {набор данных, ассоциируемый с моделью и
    определяющий то, как будет отрисована ее поверхность
    в зависимости от выбранного шейдера.}
\end{rglossary}

\paragraph{Распространенные причины проблем с производительностью}

\begin{enumerate}
    \item {
        Сложная геометрия модели. Под сложной геометрией понимается
        большое количество полигонов (граней) в полигональной сетке модели.
    }
    \item {
        Большое количество draw call'ов.
        Выполнение запроса на отрисовку выполняется центральным процессором и
        является достаточно трудоемкой операцией. За один вызов может быть
        обработана только одна полигональная сетка.
        Вполне возможна ситуация, при которой центральный процессор тратит больше времени,
        чтобы инициировать отрисовку, чем графический процессор будет ее исполнять.
    }
    \item {
        Трудоемкие шейдеры. Алгоритм шейдера может иметь
        высокую вычислительную сложность
        (например использовать долговычислимые тригонометрические функции) или
        иметь неподходящую структуру для запуска на графическом процессоре
        (например содержать многочисленные ветвления).
        Помимо этого работа шейдера может предполагать неоднократную
        отрисовку объекта для достижения определенного результата.
    }
\end{enumerate}

\paragraph{Существующие подходы}

\begin{enumerate}
    \item {
        Visibility culling (Отбор видимых полигонов).

        Visibility culling~--~это семейство алгоритмов,
        нацеленное на предотвращение вызовов отрисовки
        для объектов невидимых в кадре.%
        \cite{Cohenor2002}
        Пример различных техник отбора видимых полигонов
        показан на рисунке~\ref{figure:CullingTechniques}.
        Подобные алгоритмы являются очень эффективными,
        когда количество объектов единовременно видимых в виртуальной сцене
        значительно меньше их общего количества,
        например в замкнутых помещениях внутри крупного здания.

        \begin{figure}[ht]
            \centering
            \includegraphics[width=0.8\textwidth]
            {images/Three-types-of-visibility-culling-techniques.png}
            \caption{Техники отбора видимых полигонов.%
            \cite{Cohenor2002}}
            \label{figure:CullingTechniques}
        \end{figure}

        \begin{enumerate}
            \item {
                View-frustum culling~--~это техника отбора,
                отсекающая отрисовку объектов,
                находящихся за границами поля зрения виртуальной камеры.
            }
            \item {
                Back-face culling~--~это техника отбора,
                позволяющая избежать отрисовку геометрии,
                направленной в противоположную сторону от виртуальной камеры.
            }
            \item {
                Occlusion culling~--~это техника отбора,
                предотвращающая отрисовку геометрии,
                скрытой за другими объектами виртуальной сцены.
            }
        \end{enumerate}
    }
    \item {
        LOD (англ. Level of detail~--~уровень детализации).

        LOD~--~это оптимизационная техника,
        нацеленная на снижение используемого количества
        полигонов модели при ее удалении от виртуальной камеры.
        Для работы метода требуется создать несколько версий
        одной и той же модели с разным уровнем детализации (количеством полигонов),
        как это показано на рисунке~\ref{figure:LOD0-1}.

        \begin{figure}[ht]
            \centering
            \includegraphics[width=0.8\textwidth]{images/LOD0-1.png}
            \caption{Разные уровни детализации.%
            \cite{DocUnity}}
            \label{figure:LOD0-1}
        \end{figure}

        Перед вызовом запроса отрисовки производится выбор необходимой модели,
        то есть вблизи от виртуальной камеры будут отрисовываться
        детализированные модели (LOD0), а вдали упрощенные (LOD1-LOD3).
    }
    \item {
        Batching.

        Батчинг~--~это оптимизационная техника,
        предназначенная для снижения количества запросов отрисовки
        за счет группировки нескольких полигональных сеток.

        \begin{enumerate}
            \item {
                Батчинг может происходить динамически, если
                группируется множество простых объектов,
                с одинаковыми графическими материалами и текстурами
                (чего можно добиться с помощью создания атласа текстур).
                За счет этого можно отрисовывать несколько движущихся объектов
                за один вызов отрисовки, что имеет смысл, если продолжительность
                группировки меньше, чем затраты на многочисленные вызовы отрисовки.
            }
            \item {
                С другой стороны батчинг может быть статическим, то есть
                происходить однократно и объединять неизменяемую геометрию
                в одну полигональную сетку, позволяя достичь ещё большей производительности,
                чем при динамической группировке.
                Стоит отметить, что подобная группировка снижается
                эффективность отбора видимых полигонов, так как
                сгруппированные объекты начнут восприниматься как один.
            }
        \end{enumerate}
    }
    \item {
        Geometry instancing (Дублирование геометрии).

        Geometry instancing~--~это методика, позволяющая
        одновременно рендерить нес\-колько объектов, имеющих
        одинаковую полигональную сетку.
        Такой подход в основном используется для отрисовки
        повторяющихся фоновых объектов, таких как растительность или здания.
        Объекты могут иметь разное положение в пространстве,
        размеры, отличающиеся графические материалы.
        Дублирование геометрии значительно снижает
        количество запросов на отрисовку.
    }
\end{enumerate}

\paragraph{Выбранный подход}

При разработке прототипа было принято решение
использовать вариацию статического батчинга,
при котором полигональные сетки всех объектов одного слоя
(описано ранее в разделах~\ref{subsections:DomainModel}
и \ref{subsections:ClientServerDesign})
объединяются в одну на одном из этапов конвертации
исходной информационной модели.
В дальнейшем этот подход может быть пересмотрен,
так как несмотря на продемонстрированную эффективность
он обладает рядом недостатков,
а именно пропадает возможность взаимодействия с
отдельными объектами слоя, а также
теряется значительная часть атрибутивной информации.
