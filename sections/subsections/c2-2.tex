\subsection{Описание структуры клиентской и серверной части}

Теперь можно поподробнее рассмотреть архитектуру прототипа.
Как было сказано ранее, основное предназначение приложения~--~
презентация информационной модели участникам проекта,
не имеющим навыков моделирования,
например заказчикам, спонсорам или непосредственным пользователям здания.
Такая презентация поможет успешнее вносить
корректировки в проект еще на самых ранних стадиях,
что снизит финальные затраты на реализацию проекта.

Исходя из описанных требований можно сделать вывод о том,
что конечные пользователи приложения не станут использовать
программное обеспечение информационного моделирования.
Для них имеют значение только финальные или промежуточные результаты работы.
Поэтому было принято решение отделить функциональность приложения,
взаимодействующую с исходными данными модели.
Так же это может повысить производительность визуализации в клиентской части,
так как все трудоемкие вычисления, проводимые над исходными данными,
можно проводить на серверной части.
Наконец, клиентская часть может не зависеть
от конкретной системы информационного моделирования,
что упростит расширение системы в дальнейшем.

\paragraph{Функциональность серверной части}

\begin{enumerate}
    \item Хранение исходной информационной модели
    \item Конвертация формата RVT в формат fbx
    \item Предоптимизация модели
\end{enumerate}

\paragraph{Функциональность клиентской части}

\begin{enumerate}
    \item Загрузка готовой модели
    \item Размещение модели в виртуальной среде
    \item Взаимодействие с элементами модели
\end{enumerate}

\comment{
    TODO:

    расписать нормально пункты!
    + добавить вступление в параграфы

    хз, надо ли какие картинки или UML
    надо
}
