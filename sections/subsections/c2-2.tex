\subsection{Описание структуры клиентской и серверной части}
\label{subsections:ClientServerDesign}

Теперь можно поподробнее рассмотреть архитектуру прототипа.
Как было сказано ранее, основное предназначение приложения~--~
презентация информационной модели участникам проекта,
не имеющим навыков моделирования,
например заказчикам, спонсорам или непосредственным пользователям здания.
Такая презентация поможет успешнее вносить
корректировки в проект еще на самых ранних стадиях,
что снизит финальные затраты на реализацию проекта.%
\cite{Davidson2019}

Исходя из описанных требований можно сделать вывод о том,
что конечные пользователи приложения не станут использовать
программное обеспечение информационного моделирования.
Для них имеют значение только финальные или промежуточные результаты работы.
Поэтому было принято решение отделить функциональность приложения,
взаимодействующую с исходными данными модели.
Клиентская часть может не зависеть
от конкретной системы информационного моделирования,
что упростит расширение системы в дальнейшем.
Так же это может повысить производительность визуализации в клиентской части,
так как все трудоемкие вычисления, проводимые над исходными данными,
можно проводить на серверной части.

\paragraph{Функциональность серверной части}

Далее представлена основная функциональность реализуемая прототипом на "сервере".

\begin{itemize}
    \item {
        Хранение исходной информационной модели.
        Сервер должен хранить исходную модель,
        а также распознавать ее изменения для повторного преобразования.
    }
    \item {
        Конвертация формата RVT в формат FBX.
        Поскольку формат данных информационной модели отличается от
        представления классических форматов трехмерной графики
        и не поддерживается большинством фреймворков
        создания интерактивных графических приложений
        вроде Unity и Unreal, необходимо производить
        преобразование формата.
    } 
    \item {
        Предоптимизация модели.
        Из-за комплексности трехмерных моделей,
        получаемых после конвертации,
        необходимо проводить их дополнительную оптимизацию.
        Подробнее это будет рассмотрено в разделе~\ref{subsections:Optimization}.
    }
    \item {
        Упаковка моделей.
        Оптимизированные модели должны упаковываться для
        загрузки в клиентской части приложения.
    }
\end{itemize}

\paragraph{Функциональность клиентской части}

Далее описана функциональность клиентской части приложения
вместе с необходимыми схемами.
Функциональность, реализуемая не автором отчета,
а другими членами команды разработки, была опущена.

\begin{itemize}
    \item {
        Загрузка готовой модели.
        Приложение должно загружать выбранную упакованную модель.
        Так как в рамках прототипа не производится реализация удаленного сервера,
        загрузка будет происходить локально, как показано на рисунке~%
        \ref{figure:CModelLoader}.

        \begin{figure}[ht]
            \centering
            \includegraphics[width=0.6\textwidth]{example-image}
            \caption{Загрузчик моделей.}
            \label{figure:CModelLoader}
        \end{figure}

        \intextcomment{
            Описание UML-схемы...
        }
    } 
    \item {
        Размещение модели в виртуальной среде.
        После загрузки модель должна размещаться в виртуальной сцене.
        Для визуализации будет использоваться виртуальный стол,
        на который будет проецироваться модель здания
        в уменьшенном масштабе согласно схеме на рисунке~%
        \ref{figure:CStand}.

        \begin{figure}[ht]
            \centering
            \includegraphics[width=0.6\textwidth]{example-image}
            \caption{Размещение модели.}
            \label{figure:CStand}
        \end{figure}

        \intextcomment{
            Описание UML-схемы...
        }
    } 
    \item {
        Взаимодействие с элементами модели.
        После размещения модель будет разбивать на слои,
        представляющие отдельные структурные элементы здания,
        например стены, лестницы или двери.
        На каждый отдельный слой может накладываться эффект,
        изменяющий его отображение.
        На рисунке~\ref{figure:CLayers} показана
        структура системы слоев.

        \begin{figure}[ht]
            \centering
            \includegraphics[width=0.6\textwidth]{example-image}
            \caption{Система слоев.}
            \label{figure:CLayers}
        \end{figure}

        \intextcomment{
            Описание UML-схемы...
        }
    } 
\end{itemize}

\comment{
    TODO:

    Запилить UML!
}
