\subsection{Описание структуры клиентской и серверной части}

Теперь можно поподробнее рассмотреть архитектуру прототипа.
Как было сказано ранее, основное предназначение приложения~--~
презентация информационной модели участникам проекта,
не имеющим навыков моделирования,
например заказчикам, спонсорам или непосредственным пользователям здания.
Такая презентация поможет успешнее вносить
корректировки в проект еще на самых ранних стадиях,
что снизит финальные затраты на реализацию проекта.%
\cite{Davidson2019}

Исходя из описанных требований можно сделать вывод о том,
что конечные пользователи приложения не станут использовать
программное обеспечение информационного моделирования.
Для них имеют значение только финальные или промежуточные результаты работы.
Поэтому было принято решение отделить функциональность приложения,
взаимодействующую с исходными данными модели.
Клиентская часть может не зависеть
от конкретной системы информационного моделирования,
что упростит расширение системы в дальнейшем.
Так же это может повысить производительность визуализации в клиентской части,
так как все трудоемкие вычисления, проводимые над исходными данными,
можно проводить на серверной части.

\paragraph{Функциональность серверной части}

Далее представлена основная функциональность реализуемая прототипом на "сервере".

\begin{enumerate}
    \item {
        Хранение исходной информационной модели.
        Сервер должен хранить исходную модель,
        а также распознавать ее изменения для повторного преобразования.
    }
    \item {
        Конвертация формата RVT в формат FBX.
        Поскольку формат данных информационной модели отличается от
        представления классических форматов трехмерной графики
        и не поддерживается большинством фреймворков
        создания интерактивных графических приложений
        вроде Unity и Unreal, необходимо производить
        преобразование формата.
    } 
    \item {
        Предоптимизация модели.
        Из-за комплексности трехмерных моделей,
        получаемых после конвертации,
        необходимо проводить их дополнительную оптимизацию.
        Подробнее это будет рассмотрено в следующем разделе.
    }
    \item {
        Упаковка моделей.
        Оптимизированные модели должны упаковываться для
        загрузки в клиентской части приложения.
    }
\end{enumerate}

\paragraph{Функциональность клиентской части}

\intextcomment{
    Дополнительное описание и раскрытие пунктов списка!
}

\begin{enumerate}
    \item {
        Загрузка готовой модели.

        \begin{figure}[h]
            \centering
            \includegraphics[width=\textwidth*6/10]{example-image}
            \caption{Загрузчик моделей}
            \label{figure:ClModelLoader}
        \end{figure}
    } 
    \item {
        Размещение модели в виртуальной среде.

        \begin{figure}[h]
            \centering
            \includegraphics[width=\textwidth*6/10]{example-image}
            \caption{Размещение моделей}
            \label{figure:ClStand}
        \end{figure}
    } 
    \item {
        Взаимодействие с элементами модели

        \begin{figure}[h]
            \centering
            \includegraphics[width=\textwidth*6/10]{example-image}
            \caption{Система слоев}
            \label{figure:ClLayers}
        \end{figure}
    } 
\end{enumerate}

\comment{
    TODO:

    Запилить UML!
}
