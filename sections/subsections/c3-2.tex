\subsection{Серверная часть}

%\lipsum[5]

\paragraph{Конвертация информационной модели в FBX-формат}

%\lipsum[5]

\paragraph{Загрузка модели в промежуточный проект}

%\lipsum[5]

\paragraph{Упаковка моделей}

%\lipsum[5]

\comment{
    TODO:

    - вступление
        о том, что сервер не находится на другой машине(?)
        перечисление инструментария
    - Конвертация информационной модели в fbx
        Сказать, что это делается с помощью 3ds max
        Расписать пайплайн
        Сказать, что на этом этапе происходит основная оптимизация через батчинк
        Сказать про частичную потерю атрибутивки
        Напиздеть, что автоматизировал с помощью api
    - Загрузка модели в промежуточный проект
        Насройки импорта
        Сказать, что использовал api для извлечения атрибутивки
            можно через UML, хотя хз чего там рисовать
    - Упаковка моделей
        Рассказать про механизм AssetBundle'ов
        Показать, как создаются
            можно через UML
        Сказать, что после упаковки бандлы можно передовать по сети и кэшировать

    скрипты смотреть тут
        C:\Users\Egor\Rubius\model-import-test\Assets\Scripts\ModelImport\Editor
}
