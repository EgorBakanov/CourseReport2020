\subsection{Клиентская часть}
\label{subsections:ClientImpl}

%\lipsum[6]

\paragraph{Загрузка модели}

%\lipsum[6]

\paragraph{Размещение модели}

%\lipsum[6]

\paragraph{Слои модели}

%\lipsum[6]

\paragraph{Режим слоя: срезка}

%\lipsum[6]
.

\comment{
    TODO:

    - возможно какое-нибудь вступление
    - загрузка модели из AssetBundle'а
        скорее всего нужно запихать UML с классами 
    - размещение модели в сцене
        координатная модель Unity на трансформах
        можно добавить формулу вычисления позиции и скейла
            можно добавить UML для "public void Place(GameObject obj)" метода
            https://tfs.rubius.com/DevSaunaProjects/_git/BIMExplorer?path=%2FAssets%2FScripts%2FModelScaling%2FStand.cs&version=GBdev
    - Слои модели
        точно UML с классами
            Layer, LayerMode + несколько режимов
            https://tfs.rubius.com/DevSaunaProjects/_git/BIMExplorer?path=%2FAssets%2FScripts%2FModelLayers&version=GBdev
        описать пример работы слоев
        сказать, что для подсветки была взята готовая реализация
    - Режим слоя-срезка
        сказать про переключение шейдера
            https://tfs.rubius.com/DevSaunaProjects/_git/BIMExplorer?path=%2FAssets%2FScripts%2FModelLayers%2FCrossSectionMode.cs&version=GBdev
        общая идея шейдера
            сказать про разные версии шейдера
                прозрачный-непрозрачный
                metalic-specular
            https://tfs.rubius.com/DevSaunaProjects/_git/BIMExplorer?path=%2FAssets%2FShaders%2FLayers%2FMetalicCrossSection.shader&version=GBdev

    Добавить картинку/ки с демонстрацией переключения слоев
}
