\paragraph{Академическая среда}

В статье Джордана Дэвидсона и др. описана разработка прототипа приложения
на основе Enscape расширения для Revit, описанного раннее.%
\cite{Davidson2019}
Целью работы было расширение уже имеющихся возможностей Enscape
для экспериментальной проверки пользовательского опыта при инспектировании модели.
Приложение Дэвидсона отличается от предыдущих тем,
что обладает обратной связью с отображаемой информационной моделью,
позволяющей пользователю прямо внутри VR симуляции изменять окружение.
В рамках прототипа эта особенность ограничивалась изменением
внутреннего интерьера, мебели и характеристик окон.
Как утверждает автор, данный подход повышает вовлеченность
клиента в проект на ранних стадиях разработки,
повышает его осведомленность о решениях, принятых архитекторами,
и снижает риск изменений в ``последнюю минуту'',
повышающих стоимость реализации проекта.

Ещё одно возможное решение задачи описал в своей работе Фарзад Пур Рахимян и др.%
\cite{PourRahimian2019}
Помимо частичной обратной связи, аналогичной той,
что представлена в работе Джордана Дэвидсона,
прототип Рахимяна уникален использованием в качестве целевого формата информационных моделей
Industry Foundation Classes -- открытый, международный и
независимый от других производителей стандарт,
разработанный buildingSMART.%
\cite{BuildingSmartIFC}
В работе также описана клиент-серверная архитектура решения:
на клиентской части приложения происходит интерактивная демонстрация модели
в виртуальной или дополненной реальности;
сервер же является промежуточным слоем между исходной информационной моделью и визуализацией,
отслеживающим изменения и производящим синхронизацию данных.
