\paragraph{Слои модели}

Наконец, после успешного размещения модель разбивается на слои,
представляющие различные структурные элементы здания,
например стены, окна или лестницы. На любой выделенный слой
можно накладывать эффект отображения, как это описывалось
в разделе~\ref{subsections:ClientServerDesign}.

\dots\cite{UnityFxOutline}\dots

\comment{
    - Слои модели
        точно UML с классами
            Layer, LayerMode + несколько режимов
            https://tfs.rubius.com/DevSaunaProjects/_git/BIMExplorer?path=%2FAssets%2FScripts%2FModelLayers&version=GBdev
        описать пример работы слоев
        сказать, что для подсветки была взята готовая реализация
    - Режим слоя-срезка
        сказать про переключение шейдера
            https://tfs.rubius.com/DevSaunaProjects/_git/BIMExplorer?path=%2FAssets%2FScripts%2FModelLayers%2FCrossSectionMode.cs&version=GBdev
        общая идея шейдера
            сказать про разные версии шейдера
                прозрачный-непрозрачный
                metalic-specular
            https://tfs.rubius.com/DevSaunaProjects/_git/BIMExplorer?path=%2FAssets%2FShaders%2FLayers%2FMetalicCrossSection.shader&version=GBdev

    Добавить картинку/ки с демонстрацией переключения слоев
}