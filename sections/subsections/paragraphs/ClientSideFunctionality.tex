\paragraph{Функциональность клиентской части}

Далее описана функциональность клиентской части приложения
вместе с необходимыми схемами.
Функциональность, реализуемая не автором отчета,
а другими членами команды разработки, была опущена.

\begin{itemize}
    \item {
        Загрузка готовой модели.
        Приложение должно загружать выбранную упакованную модель.
        Так как в рамках прототипа не производится реализация удаленного сервера,
        загрузка будет происходить локально, как показано на рисунке~%
        \ref{figure:CModelLoader}.

        \begin{figure}[ht]
            \centering
            \includegraphics[width=0.6\textwidth]{example-image}
            \caption{Загрузчик моделей.}
            \label{figure:CModelLoader}
        \end{figure}

        \intextcomment{
            Описание UML-схемы...
        }
    } 
    \item {
        Размещение модели в виртуальной среде.
        После загрузки модель должна размещаться в виртуальной сцене.
        Для визуализации будет использоваться виртуальный стол,
        на который будет проецироваться модель здания
        в уменьшенном масштабе согласно схеме на рисунке~%
        \ref{figure:CStand}.

        \begin{figure}[ht]
            \centering
            \includegraphics[width=0.6\textwidth]{example-image}
            \caption{Размещение модели.}
            \label{figure:CStand}
        \end{figure}

        \intextcomment{
            Описание UML-схемы...
        }
    } 
    \item {
        Взаимодействие с элементами модели.
        После размещения модель будет разбивать на слои,
        представляющие отдельные структурные элементы здания,
        например стены, лестницы или двери.
        На каждый отдельный слой может накладываться эффект,
        изменяющий его отображение.
        На рисунке~\ref{figure:CLayers} показана
        структура системы слоев.

        \begin{figure}[ht]
            \centering
            \includegraphics[width=0.6\textwidth]{example-image}
            \caption{Система слоев.}
            \label{figure:CLayers}
        \end{figure}

        \intextcomment{
            Описание UML-схемы...
        }
    } 
\end{itemize}

\comment{
    TODO:

    Запилить UML!
}
