\paragraph{Размещение модели}

После загрузки и распаковки модель должна размещаться на виртуальном столе.
Как было сказано в разделе~\ref{subsections:ClientServerDesign},
габариты размещаемой модели должны соответствовать доступному на столе пространству.
В дополнение к этому пользователь должен иметь возможность изменять
свой размер в зависимости от масштабов загруженной модели.

В компонентной архитектуре фреймворка Unity за расположение объектов
в пространстве отвечает компонент \emph{Transform}.
Этот компонент обладает такими параметрами, как
положение (\emph{Position}) и размер (\emph{Scale}) объекта,
выражаемые трехмерными вещественными векторами (\emph{Vector3}),
а также поворот (\emph{Rotation}), выражаемый кватернионом (\emph{Quaternion}).
Компонент \emph{Transform} может исчислять свои параметры
в глобальной системе координат, либо локально
относительно другого компонента \emph{Transform}.
Компонент \emph{Transform} добавляется абсолютно каждому объекту
виртуальной сцены Unity при инстацировании объекта
и используется многими стандартными модулями,
например встроенными системами физики и отрисовки объектов.%
\cite{DocUnity}

Для размещения модели реализуется вспомогательный компонент \emph{Stand},
привязанный к объекту виртуального стола. Процесс расчета положения
и размера для загруженной модели показан на рисунке~\ref{figure:SPlaceModel}.

\begin{figure}[!htp]
    \centering
    \includegraphics[width=0.6\textwidth]{example-image}
    \caption{Размещение модели на виртуальном столе.}
    \label{figure:SPlaceModel}
    \comment{Stand.Place в качестве Stand.SetModel}
\end{figure}

\intextcomment{
    Описание UML\dots
    \comment{
        можно добавить формулу вычисления позиции и скейла
    }
}

После размещения модели пользователь может изменять свой размер
за счет вспомогательного компонента \emph{UserScaler},
использующего параметр \emph{rescaleFactor}, рассчитанный
при размещении информационной модели. Изменение размера пользователя
происходит в пределах между естественными размерами пользователя
и размерами пользователя, соответствующими масштабу размещенной модели,
что соответствует алгоритму на схеме~\ref{figure:SSetUserSize}.

\begin{figure}[!htp]
    \centering
    \includegraphics[width=0.6\textwidth]{example-image}
    \caption{Размещение модели на виртуальном столе.}
    \label{figure:SSetUserSize}
    \comment{Scaler.UpdateSize в качестве UserScaler.SetSize}
\end{figure}

\intextcomment{
    Описание UML\dots
}
Пример размещенной на виртуальном столе модели здания отеля
показан на рисунке~\ref{figure:PlacedModelExample}.

\begin{figure}[!htp]
    \centering
    \includegraphics[width=0.6\textwidth]{example-image}
    \caption{Пример размещенной на виртуальном столе
    информационной модели здания.}
    \label{figure:PlacedModelExample}
    \comment{Скриншот}
\end{figure}
