\paragraph{Размещение модели}

После загрузки и распаковки модель должна размещаться на виртуальном столе.
Как было сказано в разделе~\ref{subsections:ClientServerDesign},
габариты размещаемой модели должны соответствовать доступному на столе месту.
В дополнение к этому пользователь должен иметь возможность изменять
свой размер в зависимости от масштабов загруженной модели.

\begin{figure}[!htp]
    \centering
    \includegraphics[width=0.6\textwidth]{example-image}
    \caption{Размещение модели на виртуальном столе.}
    \label{figure:SPlaceModel}
    \comment{Stand.Place в качестве Stand.SetModel}
\end{figure}

\intextcomment{
    Описание UML\dots
}

\begin{figure}[!htp]
    \centering
    \includegraphics[width=0.6\textwidth]{example-image}
    \caption{Размещение модели на виртуальном столе.}
    \label{figure:SSetUserSize}
    \comment{Scaler.UpdateSize в качестве UserScaler.SetSize}
\end{figure}

\intextcomment{
    Описание UML\dots
}

\begin{figure}[!htp]
    \centering
    \includegraphics[width=0.6\textwidth]{example-image}
    \caption{Пример размещенной на виртуальном столе
    информационной модели здания.}
    \label{figure:PlacedModelExample}
    \comment{Скриншот}
\end{figure}

\comment{
    - размещение модели в сцене
    координатная модель Unity на трансформах
    можно добавить формулу вычисления позиции и скейла
}