\subsection{Требования к системе}
\label{subsections:Requirements}

Далее приведен формализованный список требований к разрабатываемому прототипу.

\paragraph{Нефункциональные требования}
\begin{itemize}
    \item совместимость с форматом Revit,
    как наиболее распространенной системы информационного моделирования;
    \item разделение функциональности приложения на ``серверную'' и ``клиентскую''
    для повышения гибкости системы к возможным дальнейшим расширениям и
    независимости пользователя от конкретных систем информационного моделирования;
    \item высокая производительность при инспектировании комплексных моделей;
    \item извлечение атрибутивной информации,
    закладываемой в модель.
\end{itemize}

Стоит отметить, что в рамках прототипа не производилась
реализация полноценного удаленного сервера,
а производилась эмуляция его работы локально.

\paragraph{Функциональные требования}
\begin{itemize}
    \item выбор BIM модели из уже загруженных на сервер;
    \item изменение масштаба отображения модели
    между натуральной величиной и размерами настольного макета;
    \item управление отображением различных слоев модели,
    созданных на основе извлеченной атрибутивной информации;
\end{itemize}

Особенности реализации пользовательского интерфейса и
функциональности навигации в виртуальном пространстве будут опущены,
так как реализовывались другими членами команды разработки прототипа.