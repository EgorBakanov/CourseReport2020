\unorderedsection{Глоссарий}

\begin{rglossary}
    \rglossaryline{Unity}
    {игровой фреймворк, используемый для трехмерной визуализации.%
    \cite{DocUnity}}

    \rglossarylineExpl{BIM}
    {англ. Building Information Model}
    {цифровой проект здания или другого объекта инфраструктуры,
    сопровождаемый базой данных всех его физических и функциональных характеристик.%
    \cite{NationalBIMfaqs}}

    \rglossarylineExpl{Шейдер}
    {англ. shader}
    {разновидность компьютерных программ, запускаемых на графических процессорах,
    предназначенных для отрисовки изображений.}

    % возможно стоит убрать
    \rglossaryline{HTC Vive}
    {шлем виртуальной реальности, разрабатываемый компаниями HTC и Valve.}

    \rglossarylineExpl{Фреймворк}
    {англ. framework}
    {переиспользуемая, "незавершенная" система,
    которая может использоваться для создания другой производной системы.%
    \cite{Johnson1988,Schmidt2000}}

    \rglossarylineExpl{Меш или полигональная сетка}
    {англ. polygon mesh}
    {структура данных, содержащая набор вершин, ребер и граней,
    определяющих форму многогранного объекта.}

    \rglossarylineExpl{Рендеринг или отрисовка}
    {англ. rendering}
    {процесс получения изображения двумерной или трехмерной модели
    с помощью компьютерной программы.}
\end{rglossary}